\documentclass{article}
\usepackage[utf8]{inputenc}
\usepackage{geometry}
\usepackage{xcolor}
\usepackage{makecell}
\usepackage[T1]{fontenc}
\usepackage[utf8]{inputenc}
\usepackage{newunicodechar}
\newunicodechar{ }{\,}
\geometry{a4paper, margin=1in}
\begin{document}
\tableofcontents
\newpage\section{Reden}\subsection{Redeinformationen ID208303900}
\begin{itemize}
\item Redner: Beate Walter-Rosenheimer
\item Fraktion: BÜNDNIS 90/DIE GRÜNEN
\item Wahlperiode: 20
\item Sitzungsnummer: 83
\item Rede ID: ID208303900
\item Datum der Sitzung: 27.01.2023
\item Tagesordnungspunkt: Zusatzpunkt 8
\item Tagesordnungspunkt Beschreibung: Beratung des Antrags der Abgeordneten Jürgen Braun, Martin Sichert, Marcus Bühl, weiterer Abgeordneter und der Fraktion der AfD

\end{itemize}
\subsubsection{Redetext}
\definecolor{CustomColor0}{RGB}{255,255,255}
\colorbox{CustomColor0}{\parbox{\linewidth}{Frau Präsidentin!}}

\colorbox{CustomColor0}{\parbox{\linewidth}{Sehr geehrte Kollegen und Kolleginnen!}}

\definecolor{CustomColor1}{RGB}{71,255,71}
\colorbox{CustomColor1}{\parbox{\linewidth}{Liebe Zuhörerinnen und Zuhörer!}}

\colorbox{CustomColor0}{\parbox{\linewidth}{Seid Menschen!}}

\definecolor{CustomColor2}{RGB}{255,163,163}
\colorbox{CustomColor2}{\parbox{\linewidth}{Menschen haben es getan, weil sie Menschen nicht als Menschen anerkannt haben.}}

\definecolor{CustomColor3}{RGB}{221,255,221}
\colorbox{CustomColor3}{\parbox{\linewidth}{Man kann nicht alle Menschen lieben, aber Respekt gebührt jedem.}}

\definecolor{CustomColor4}{RGB}{152,255,152}
\colorbox{CustomColor4}{\parbox{\linewidth}{Es gibt kein christliches, kein jüdisches, kein muslimisches Blut, es gibt nur menschliches Blut.}}

\colorbox{CustomColor0}{\parbox{\linewidth}{Wir sind alle gleich.}}

\definecolor{CustomColor5}{RGB}{89,255,89}
\colorbox{CustomColor5}{\parbox{\linewidth}{Das ist – mit Erlaubnis der Präsidentin – ein Zitat der Holocaustüberlebenden Margot Friedländer, die heute bei unserer Gedenkstunde auch zu Gast war und vor der ich mich verneigen möchte.}}

\colorbox{CustomColor0}{\parbox{\linewidth}{Mit diesem Zitat ist eigentlich alles zu diesem Antrag gesagt,(Beifall}}

\definecolor{CustomColor6}{RGB}{127,255,127}
\colorbox{CustomColor6}{\parbox{\linewidth}{beim BÜNDNIS 90/DIE GRÜNEN, bei der SPD und der FDP sowie bei Abgeordneten der CDU/CSU)einem Antrag, der einmal mehr gegen Muslime hetzt unter dem Vorwand des Schutzes der Christen auf der Welt.}}

\definecolor{CustomColor7}{RGB}{133,255,133}
\colorbox{CustomColor7}{\parbox{\linewidth}{Auf diesen Schutz können wir Christinnen und Christen gut verzichten.}}

\colorbox{CustomColor0}{\parbox{\linewidth}{Die Christinnen und Christen brauchen Sie dazu bestimmt nicht.}}

\definecolor{CustomColor8}{RGB}{228,255,228}
\colorbox{CustomColor8}{\parbox{\linewidth}{Die weltweite Verfolgung von Christinnen und Christen ist in der Tat ein riesiges Problem, die Tendenz ist steigend.}}

\definecolor{CustomColor9}{RGB}{173,255,173}
\colorbox{CustomColor9}{\parbox{\linewidth}{Es ist aber unverschämt, die Verfolgung von Christen weltweit zu nutzen, um hier die altbekannte Muslimenhetze, Islamhetze vorzubringen.}}

\definecolor{CustomColor10}{RGB}{136,255,136}
\colorbox{CustomColor10}{\parbox{\linewidth}{Um nichts anderes handelt es sich in diesem Antrag.(Beifall beim BÜNDNIS 90/DIE GRÜNEN und bei der SPD sowie bei Abgeordneten der CDU/CSU und des Abg.}}

\colorbox{CustomColor0}{\parbox{\linewidth}{Peter Heidt [FDP])Ob und inwieweit Christen und Angehörige anderer religiöser Minderheiten verfolgt werden}}

\colorbox{CustomColor0}{\parbox{\linewidth}{ – wir haben es heute schon gehört –, hängt nicht von den Religionen ab, sondern vom politischen System.}}

\definecolor{CustomColor11}{RGB}{78,255,78}
\colorbox{CustomColor11}{\parbox{\linewidth}{Die Freiheit, einer Weltanschauung oder Religion anzugehören oder eben nicht, ist ein bereits in der Allgemeinen Erklärung der Menschenrechte der UNO von 1948 niedergelegtes grundlegendes Menschenrecht.}}

\definecolor{CustomColor12}{RGB}{147,255,147}
\colorbox{CustomColor12}{\parbox{\linewidth}{Und um das umzusetzen, braucht es Demokratien, die das festschreiben und sich um die Umsetzung kümmern.}}

\colorbox{CustomColor7}{\parbox{\linewidth}{Wir wissen:}}

\definecolor{CustomColor13}{RGB}{255,197,197}
\colorbox{CustomColor13}{\parbox{\linewidth}{In autokratischen Regimen werden Menschenrechte und damit auch die Religionsfreiheit gezielt und systematisch verletzt.}}

\definecolor{CustomColor14}{RGB}{255,185,185}
\colorbox{CustomColor14}{\parbox{\linewidth}{Mit der Zunahme totalitärer Herrschaftssysteme haben wir auch mehr Verfolgung von Religionen.}}

\colorbox{CustomColor0}{\parbox{\linewidth}{Das ist ein Fakt.}}

\definecolor{CustomColor15}{RGB}{255,203,203}
\colorbox{CustomColor15}{\parbox{\linewidth}{Wir haben heute auch schon die Beispiele gehört: von China, wo im Namen des Staatsatheismus zur Christenverfolgung aufgerufen wird, aber auch zur Verfolgung von Muslimen, oder von Indien, wo die hindu-nationalistische Regierung massive Einschränkungen der Religionsfreiheit auch gegen Muslime, auch gegen Christen vornimmt.}}

\definecolor{CustomColor16}{RGB}{255,120,120}
\colorbox{CustomColor16}{\parbox{\linewidth}{Und wir schauen nach Myanmar, wo Christinnen und Christen verfolgt werden, Kirchen zerstört werden, und zwar von Buddhisten.}}

\colorbox{CustomColor14}{\parbox{\linewidth}{Es gibt all das, und all das ist falsch.}}

\colorbox{CustomColor0}{\parbox{\linewidth}{Der Antrag, den Sie hier einbringen, ignoriert das komplett.}}

\definecolor{CustomColor17}{RGB}{255,86,86}
\colorbox{CustomColor17}{\parbox{\linewidth}{Er ignoriert, dass jede Religion und jede Weltanschauung dazu missbraucht werden kann, gegen Andersdenkende und Angehörige von Minderheiten zu agieren.}}

\definecolor{CustomColor18}{RGB}{252,255,252}
\colorbox{CustomColor18}{\parbox{\linewidth}{Dazu gibt es auch in der Geschichte des Christentums genug erschreckende Beispiele.\textbf{(Beifall beim BÜNDNIS 90/DIE GRÜNEN sowie bei Abgeordneten der SPD und der FDP)}}}

\definecolor{CustomColor19}{RGB}{179,255,179}
\colorbox{CustomColor19}{\parbox{\linewidth}{Um das große Unrecht weltweit für alle verfolgten Religionen, für alle Minderheiten, für alle Menschen zu beenden, reicht ein Gedenktag mit Sicherheit nicht aus.}}

\colorbox{CustomColor0}{\parbox{\linewidth}{Wir lehnen Ihren Antrag ab.}}

\colorbox{CustomColor10}{\parbox{\linewidth}{Wir lehnen Rassismus und Hetze ab und geben ihnen keinen Raum.\textbf{\textbf{(Beifall beim BÜNDNIS 90/DIE GRÜNEN, bei der SPD und der FDP sowie bei Abgeordneten der CDU/CSU)}}}}

\colorbox{CustomColor0}{\parbox{\linewidth}{Für die CDU/CSU-Fraktion spricht nun Dr. Jonas Geissler.\textbf{(Beifall bei der CDU/CSU)}}}

\subsubsection*{POS Statistics}
\begin{tabular}{|l|c|}
\hline
\textbf{POS Coarse Value} & \textbf{Count} \\
\hline
CCONJ & 35 \\
\hline
AUX & 24 \\
\hline
PRON & 42 \\
\hline
PROPN & 29 \\
\hline
ADJ & 22 \\
\hline
NUM & 5 \\
\hline
SCONJ & 9 \\
\hline
ADP & 57 \\
\hline
DET & 75 \\
\hline
ADV & 40 \\
\hline
PUNCT & 77 \\
\hline
PART & 9 \\
\hline
VERB & 42 \\
\hline
NOUN & 114 \\
\hline
\end{tabular}
\subsubsection*{Named Entity Statistics}
\begin{tabular}{|l|c|}
\hline
\textbf{Named Entity Value} & \textbf{Count} \\
\hline
LOC & 4 \\
\hline
ORG & 18 \\
\hline
MISC & 12 \\
\hline
PER & 4 \\
\hline
\end{tabular}
\subsubsection*{Topic Statistics}
\begin{tabular}{|l|c|}
\hline
\textbf{Topic Value} & \textbf{Mean Score} \\
\hline
Foreign & 3.710963647951276E-4 \\
\hline
Social & 0.029063461395708146 \\
\hline
Law & 0.00508407638859781 \\
\hline
Health & 0.001609036519579711 \\
\hline
Government & 0.2607249823771888 \\
\hline
Transportation & 7.877580033249722E-4 \\
\hline
Public & 4.5607055479943967E-4 \\
\hline
Housing & 0.001502764976789074 \\
\hline
Labor & 0.0012929970925517195 \\
\hline
Defense & 0.001240438510729505 \\
\hline
Civil & 0.45465661978232674 \\
\hline
Energy & 7.820966578251767E-4 \\
\hline
Education & 8.74536817183766E-4 \\
\hline
Technology & 0.0014083790049827418 \\
\hline
Domestic & 0.004178828867091344 \\
\hline
Immigration & 0.019010105667669395 \\
\hline
Environment & 0.0026219112684002003 \\
\hline
Agriculture & 5.767418274712434E-4 \\
\hline
Macroeconomics & 0.007979918412141107 \\
\hline
Culture & 9.40625100316966E-4 \\
\hline
International & 0.2048375395152909 \\
\hline
\end{tabular}
\subsection{Redeinformationen ID202818100}
\begin{itemize}
\item Redner: Beate Walter-Rosenheimer
\item Fraktion: BÜNDNIS 90/DIE GRÜNEN
\item Wahlperiode: 20
\item Sitzungsnummer: 28
\item Rede ID: ID202818100
\item Datum der Sitzung: 07.04.2022
\end{itemize}
\subsubsection{Redetext}
\definecolor{CustomColor0}{RGB}{255,98,98}
\colorbox{CustomColor0}{\parbox{\linewidth}{Bei diesem Thema gibt es für mich keine einfache und schnelle Antwort, und ich habe alle Argumente immer wieder sehr ernsthaft gegeneinander abgewogen.}}

\definecolor{CustomColor1}{RGB}{255,162,162}
\colorbox{CustomColor1}{\parbox{\linewidth}{Meine Einschätzung ist auch abhängig von der jeweiligen pandemischen Lage.}}

\definecolor{CustomColor2}{RGB}{235,255,235}
\colorbox{CustomColor2}{\parbox{\linewidth}{Grundsätzlich bin ich nicht dafür, dass der Staat so weit in die Persönlichkeit des Einzelnen eingreift.}}

\definecolor{CustomColor3}{RGB}{141,255,141}
\colorbox{CustomColor3}{\parbox{\linewidth}{Denn eine Impfpflicht stellt zweifelsohne einen Eingriff in das Grundrecht der körperlichen Unversehrtheit nach Artikel 2 Absatz 2 Satz 1 GG dar.}}

\definecolor{CustomColor4}{RGB}{191,255,191}
\colorbox{CustomColor4}{\parbox{\linewidth}{Einschränkungen dieses Grundrechts müssen gut begründet sein.}}

\definecolor{CustomColor5}{RGB}{255,46,46}
\colorbox{CustomColor5}{\parbox{\linewidth}{Seit Beginn der Pandemie hatten wir zahlreiche schwere und schwerste Verläufe in den Krankenhäusern und auf den Intensivstationen, viele Tausend Tote, das Pflegepersonal und die Ärzt/-innen waren am Belastungslimit.}}

\definecolor{CustomColor6}{RGB}{255,157,157}
\colorbox{CustomColor6}{\parbox{\linewidth}{Die Einschränkungen zur Pandemieeindämmung waren für alle hart.}}

\definecolor{CustomColor7}{RGB}{255,255,255}
\colorbox{CustomColor7}{\parbox{\linewidth}{Lockdowns, Kontaktverbote, Ausgehbeschränkungen usw.}}

\definecolor{CustomColor8}{RGB}{255,34,34}
\colorbox{CustomColor8}{\parbox{\linewidth}{Die Verfügbarkeit der zugelassenen Impfstoffe schien ein Ausweg aus den ständigen Lockdowns, aus den Kita- und Schulschließungen, der Überlastung der Eltern und dem Leid, der Einsamkeit in Senioren- und Pflegeheimen zu sein.}}

\definecolor{CustomColor9}{RGB}{215,255,215}
\colorbox{CustomColor9}{\parbox{\linewidth}{Und aus den schweren psychosozialen, wirtschaftlichen und gesellschaftlichen Folgen der andauernden Maßnahmen.}}

\definecolor{CustomColor10}{RGB}{185,255,185}
\colorbox{CustomColor10}{\parbox{\linewidth}{Und die Impfung hat sich durchaus als wirksam erwiesen.}}

\definecolor{CustomColor11}{RGB}{92,255,92}
\colorbox{CustomColor11}{\parbox{\linewidth}{Ich habe mich gern impfen und auch boostern lassen und möchte alle Menschen ermutigen, diesen Schritt zu gehen.}}

\definecolor{CustomColor12}{RGB}{255,58,58}
\colorbox{CustomColor12}{\parbox{\linewidth}{Denn die Impfung schützt in aller Regel vor schweren Verläufen und Tod und minimiert die Einweisungen in ein Krankenhaus oder auf eine Intensivstation.}}

\definecolor{CustomColor13}{RGB}{255,241,241}
\colorbox{CustomColor13}{\parbox{\linewidth}{So trägt die Impfung zur Entlastung der angespannten medizinischen Situation bei.}}

\colorbox{CustomColor9}{\parbox{\linewidth}{Aus diesem Grund habe ich mich seit einigen Monaten, als klar war, dass die Impfquote in Deutschland eher niedrig sein würde, für eine allgemeine Impfpflicht ab 18 Jahren ausgesprochen.}}

\definecolor{CustomColor14}{RGB}{255,174,174}
\colorbox{CustomColor14}{\parbox{\linewidth}{Auch heute ist die Impfquote noch deutlich geringer als in vielen unserer Nachbarländer.}}

\definecolor{CustomColor15}{RGB}{255,248,248}
\colorbox{CustomColor15}{\parbox{\linewidth}{Von einer möglichst hohen Impfquote versprach ich mir ein baldiges Ende der Maßnahmen für alle und eine Rückkehr zur „Normalität“.Omikron hat für mich die Situation verändert und sich als Gamechanger erwiesen.}}

\colorbox{CustomColor15}{\parbox{\linewidth}{Die Verläufe sind milder, es erkranken weniger Menschen schwer und es müssen weniger Patient/-innen auf Intensivstationen verlegt werden.}}

\definecolor{CustomColor16}{RGB}{255,185,185}
\colorbox{CustomColor16}{\parbox{\linewidth}{Viele Argumente für eine Impfpflicht wurden durch Omikron entkräftet.}}

\definecolor{CustomColor17}{RGB}{255,193,193}
\colorbox{CustomColor17}{\parbox{\linewidth}{Denn auch Geimpfte können andere infizieren, da alle zugelassenen Impfstoffe gegen SARS-CoV‑2 keine sterile Immunität gewährleisten.}}

\colorbox{CustomColor10}{\parbox{\linewidth}{Dennoch schützen die Impfungen gut vor schweren Erkrankungen, das will ich betonen.}}

\colorbox{CustomColor7}{\parbox{\linewidth}{Dazu gibt es mittlerweile valide Studienergebnisse.}}

\colorbox{CustomColor7}{\parbox{\linewidth}{Die Impfpflicht müsste darüber hinaus eine drittschützende Wirkung entfalten.}}

\colorbox{CustomColor7}{\parbox{\linewidth}{Die staatliche Schutzpflicht bezieht sich ausdrücklich nicht auf eine Selbstgefährdung.}}

\definecolor{CustomColor18}{RGB}{220,255,220}
\colorbox{CustomColor18}{\parbox{\linewidth}{Die wissenschaftlichen Erkenntnisse, die für eine Erfüllung der Schutzpflicht des Staates durch eine Impfpflicht sprechen, sind – allein schon aufgrund einer noch nicht ausreichenden Datenlage – bisher noch unklar.}}

\definecolor{CustomColor19}{RGB}{255,242,242}
\colorbox{CustomColor19}{\parbox{\linewidth}{Außerdem sehe ich eine deutliche Erhöhung der Impfquote, aufgrund einer Impfpflicht ab 60 Jahre, als nicht gesichert an.}}

\definecolor{CustomColor20}{RGB}{255,170,170}
\colorbox{CustomColor20}{\parbox{\linewidth}{Aus diesen Gründen halte ich eine Impfpflicht zum jetzigen Zeitpunkt – und vorbehaltlich der noch unklaren Prognosen für den nächsten Herbst – für nicht geeignet.}}

\definecolor{CustomColor21}{RGB}{255,197,197}
\colorbox{CustomColor21}{\parbox{\linewidth}{Ich schließe sie aber ausdrücklich nicht für alle Zeiten aus.}}

\definecolor{CustomColor22}{RGB}{120,255,120}
\colorbox{CustomColor22}{\parbox{\linewidth}{Und ich weise noch einmal darauf hin:}}

\definecolor{CustomColor23}{RGB}{157,255,157}
\colorbox{CustomColor23}{\parbox{\linewidth}{Die Impfungen sind wirkungsvoll.}}

\colorbox{CustomColor7}{\parbox{\linewidth}{Deshalb appelliere ich an alle Menschen, sich impfen zu lassen.}}

\definecolor{CustomColor24}{RGB}{63,255,63}
\colorbox{CustomColor24}{\parbox{\linewidth}{Ich danke meiner Fraktion für die Möglichkeit einer ergebnisoffenen und konstruktiven Debatte.}}

\definecolor{CustomColor25}{RGB}{53,255,53}
\colorbox{CustomColor25}{\parbox{\linewidth}{Ich bin froh, dass diese Abstimmung als Gewissensentscheidung anerkannt wurde.}}

\subsubsection*{POS Statistics}
\begin{tabular}{|l|c|}
\hline
\textbf{POS Coarse Value} & \textbf{Count} \\
\hline
PRON & 34 \\
\hline
CCONJ & 25 \\
\hline
AUX & 28 \\
\hline
PROPN & 5 \\
\hline
ADJ & 36 \\
\hline
NUM & 6 \\
\hline
SCONJ & 6 \\
\hline
ADP & 57 \\
\hline
DET & 84 \\
\hline
ADV & 57 \\
\hline
PUNCT & 60 \\
\hline
PART & 9 \\
\hline
VERB & 41 \\
\hline
X & 2 \\
\hline
NOUN & 108 \\
\hline
\end{tabular}
\subsubsection*{Named Entity Statistics}
\begin{tabular}{|l|c|}
\hline
\textbf{Named Entity Value} & \textbf{Count} \\
\hline
LOC & 4 \\
\hline
ORG & 2 \\
\hline
MISC & 2 \\
\hline
PER & 1 \\
\hline
\end{tabular}
\subsubsection*{Topic Statistics}
\begin{tabular}{|l|c|}
\hline
\textbf{Topic Value} & \textbf{Mean Score} \\
\hline
Foreign & 3.2444126141358123E-4 \\
\hline
Social & 0.011123676920485168 \\
\hline
Law & 0.010561979722159307 \\
\hline
Health & 0.595614533567586 \\
\hline
Government & 0.1159360634968228 \\
\hline
Transportation & 8.907697980854815E-4 \\
\hline
Public & 3.272792253648566E-4 \\
\hline
Labor & 0.0017483651456793516 \\
\hline
Housing & 0.0011488416545916787 \\
\hline
Defense & 0.0014019286802177896 \\
\hline
Civil & 0.15024955680807275 \\
\hline
Energy & 6.41700393509492E-4 \\
\hline
Education & 0.02223046787854731 \\
\hline
Technology & 0.006839097313125173 \\
\hline
Domestic & 0.009359764495236574 \\
\hline
Immigration & 0.0019645053333519945 \\
\hline
Environment & 0.001969786286887635 \\
\hline
Agriculture & 0.021676441651917146 \\
\hline
Macroeconomics & 0.01112330269136168 \\
\hline
Culture & 8.977260543664064E-4 \\
\hline
International & 0.033969764210103844 \\
\hline
\end{tabular}
\end{document}